\documentclass[12pt, titlepage]{article}
\usepackage[utf8]{inputenc}
\usepackage{hyperref}
\hypersetup{
    colorlinks=true,
    linkcolor=blue,
    filecolor=magenta,      
    urlcolor=blue,
    pdftitle={Overleaf Example},
    }
\usepackage{minted}

\title{Bottom Penguin Coin - White Paper}
\author{\\\href{mailto:coins@andrewli.site}{coins@andrewli.site}\\\href{https://andrewli.site/}{andrewli.site}\\June 2021}
\date{}

\begin{document}

\maketitle

\section{Introduction}
Too long have penguins been bartering fish in the chilly condition of the Antarctica. This proposed solution will revolutionize the way we penguins live in the world with giant metal birds that float in the sky. This proposed solution involves creating these Bottom Penguin Coins with penguins, ie. out of our bottoms! 

\section{Research}
Some equations:
$$
\pi = e = 3
$$
As we can see given this, bottom penguin will be the next cryptocurrency that will take over the world. In fact if we type this in C we get $\pi = 3$ and $e = 3$, making it a certainty that Bottom Penguin Coin will take over
\begin{minted}[xleftmargin=20pt,breaklines,mathescape,
 numbersep=5pt,
 frame=single,
 numbersep=5pt,
 xleftmargin=0pt,]{c}
#include <stdio.h>

int main() {
    int pi, e;
    pi = e = 3;
    printf("pi = %d\n", pi);
    printf("e = %d\n", e);
    
    return(0);
}
\end{minted}
\newpage

\section{References}
\begin{enumerate}
  \item Some engineer
  \item Dude just trust me
  \item \href{https://github.com/Zeyu-Li/bottom-penguin}{GitHub Code}
\end{enumerate}


\end{document}
